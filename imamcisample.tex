\documentclass{amsart}
\usepackage{amsmath}
\usepackage[ocgcolorlinks,linktoc=all]{hyperref}
\hypersetup{citecolor=blue,linkcolor=red}
\newtheorem{theorem}{Theorem}
\newtheorem*{thmA}{Theorem}
\newtheorem*{thmB}{Theorem}
\newtheorem*{rem}{Remark}
\newtheorem*{thmmain}{Theorem}
\newtheorem{lemma}[theorem]{Lemma}
\newtheorem{proposition}[theorem]{Proposition}
\newtheorem*{propmain}{Proposition}
\newtheorem{corollary}[theorem]{Corollary}
\theoremstyle{definition}
\newtheorem{definition}[theorem]{Definition}
\newtheorem{example}[theorem]{Example}
\newtheorem{xca}[theorem]{Exercise}
\theoremstyle{remark}
\newtheorem{remark}[theorem]{Remark}

\newcommand{\abs}[1]{\lvert#1\rvert}
\numberwithin{equation}{section}

\newcommand{\blankbox}[2]{%
  \parbox{\columnwidth}{\centering
    \setlength{\fboxsep}{0pt}%
    \fbox{\raisebox{0pt}[#2]{\hspace{#1}}}%
  }%
}

\DeclareMathOperator{\RR}{\mathbb{R}}
\DeclareMathOperator{\sphere}{\mathbb{S}}
\DeclareMathOperator{\basepoint}{p_0}
\DeclareMathOperator{\posvec}{U}
\DeclareMathOperator{\scaledposvec}{\tilde{U}}
\DeclareMathOperator{\tangposvec}{V}
\DeclareMathOperator{\scaledtangposvec}{\tilde{V}}
\DeclareMathOperator{\radialdistance}{\rho}
\newcommand{\normalprojection}[1][]{\ensuremath{\pi^{\perp}_{#1}}}
\DeclareMathOperator{\vecflow}{\gamma}
% \DeclareMathOperator{\weingarten}{\mathcal{W}}

\DeclareMathOperator{\reflectionvector}{\vec{V}}
\newcommand{\reflectionplane}[1][\reflectionvector]{\ensuremath{P_{#1}}}
\newcommand{\reflectionmap}[1][\reflectionvector]{\ensuremath{R_{#1}}}
\newcommand{\reflectionset}[2][\reflectionvector]{\ensuremath{{#2}_{#1}}}
\newcommand{\reflectionhalfspace}[1][\reflectionvector]{\ensuremath{\reflectionset[{#1}]{H}}}
\newcommand{\vertvec}{e}
\DeclareMathOperator{\origin}{O}
\DeclareMathOperator{\radialprojection}{\pi}
\DeclareMathOperator{\height}{h}
\DeclareMathOperator{\equator}{E}
\newcommand{\ip}[2]{\ensuremath{\langle{#1},{#2}\rangle}}
\DeclareMathOperator{\intersect}{\cap}
\DeclareMathOperator{\nor}{\nu}


\begin{document}

\title[Harnack estimate for mean curvature flow on the sphere]
{Harnack estimate for mean curvature flow on the sphere}
\author[P. Bryan]{Paul Bryan}
\address{Department of Mathematics, University of California San Diego, La Jolla, USA}
\curraddr{}
\email{pabryan@gmail.com; pbryan@ucsd.edu}
\author[M.N. Ivaki]{Mohammad N. Ivaki}
\address{Institut f\"{u}r Diskrete Mathematik und Geometrie, Technische Universit\"{a}t Wien,
Wiedner Hauptstr. 8--10, 1040 Wien, Austria}
\curraddr{}
\email{mohammad.ivaki@tuwien.ac.at}
\date{\today}
\keywords{Mean Curvature Flow, Ancient solutions, Aleksandrov reflection, Harnack estimate}
\subjclass[2010]{53C44, 35K55, 58J35}

\begin{abstract}
We consider the evolution of hypersurfaces on the unit sphere $\mathbb{S}^{n+1}$ by their mean curvature. We prove a differential Harnack inequality for any weakly convex solution to the mean curvature flow. As an application, by applying an Aleksandrov reflection argument, we classify convex, ancient solutions of the mean curvature flow on the sphere.
\end{abstract}
\maketitle
We study convex hypersurfaces $M_0$, $n\ge2$, without boundary, which are smoothly embedded in $\mathbb{S}^{n+1}.$ Let $F_0:M^n\to \mathbb{S}^{n+1}$ be a smooth embedding of $M_0$. The mean curvature flow with initial data $F_0$ is a family of hypersurfaces given by embeddings $F:M^n\times[0,T)\to \mathbb{S}^{n+1}$, $~ F(\cdot,0)=F_0(\cdot)$ that moves in the direction of the unit normal vector $\nu$ with speed equal to the mean curvature $H$, the trace of the second fundamental form $A(V,V)$ over the tangent vectors $V$, e.q.,
\[\partial_tF(\cdot,t)=-H(\cdot,t)\nu(\cdot,t).\]
Here, $H(p,t)$ is the mean curvature of the hypersurface $M_t:=F(M^n,t)$ at the point $F(p,t)$ where the unit outer normal vector is $\nu(p,t).$
Let $D$ denote the Levi-Civita connection of $\mathbb{S}^{n+1}.$
\begin{thmmain}[Harnack estimate]
Suppose $M_t$ is a smooth, weakly convex solution of the mean curvature flow on the time interval $[0,T)$. For $t>0$ we have
\[\partial_tH+\frac{H}{2t}+2D_VH+A(V,V)-nH\geq 0\]
for all tangent vectors $V$.
\end{thmmain}
Comparing this with Hamilton's Harnack inequality \cite{Hamilton 95} for the mean curvature flow in $\mathbb{R}^{n+1}$, here we gained a bonus term $-nH$ through the ambient space $\mathbb{S}^{n+1}.$ This allows us,
by employing a parabolic Aleksandrov reflection argument first developed in \cite{ Chow 97,Chow-Gul 01, Chow-Gul 96},  a rather convenient classification of ancient, smooth, weakly convex solutions of the mean curvature flow.
\begin{thmmain}[Classification of ancient solutions]
The only ancient, weakly convex, embedded solutions of mean curvature on the sphere are equators and shrinking geodesic spheres.
\end{thmmain}

Let us note that such a classification has been obtained already in \cite[Theorem 6.1]{Hu-Sin 2014}, with a relatively short proof applying the maximum principle to the quantity $(\|A\|^2 - \tfrac{1}{n}H^2)/H^2$. As usual for such arguments, the Codazzi equation is employed; therefore, the result pertains to $n\geq 2$. Bryan and Louie \cite{Br-Lou} used a completely different argument to obtain the classification for $n=1$. First, a Harnack inequality is obtained, which ensures that the backwards limit as $t \to -\infty$ is an equator. Then the Aleksandrov reflection argument is employed to show maximal reflection symmetry, immediately resulting in the classification. Our work here extends the techniques of \cite{Br-Lou} from $n=1$ to $n\geq 2$ formally following the same procedure of obtaining the Harnack inequality and then using the Aleksandrov reflection to show the maximal reflection symmetry. We believe, that unlike the methods in \cite{Hu-Sin 2014}, our argument is more broadly applicable: once the Harnack is obtained for a curvature flow, the remainder of the argument holds with very little modification. This theme will be explored for a range of curvature flows similar to those studied in \cite{Andrews 94} in a forthcoming paper. In \cite{Andrews 94}, the Gauss map parametrization and support function are used to great effect, considerably simplifying the computations. A genuine difficulty on the sphere is obtaining a similar useful parametrization under which we can perform the calculations.

Before moving on to the details, a few words about the broader context are in order. It is well known that ancient solutions arise as singularity models for curvature flows (e.g. \cite{Hu-Sin 1999}); therefore, their study is of great importance. On the unit sphere, as opposed to the Euclidean case studied in \cite{Hu-Sin 2014}, where classification can be obtained only with additional assumptions such as pinching, we find a complete classification and strong rigidity. The difference is the compactness and positive curvature of the sphere. We might conjecture that for the mean curvature flow in a positively curved, compact background, there exists at most one non-trivial ancient solution emanating from each closed minimal hypersurface. However, in general, it may be too much to hope that such ancient solutions exist.

Classifications have been obtained for the curve shortening flow in the plane \cite{Da-Ham-Sesu 2010} and the Ricci flow of surfaces \cite{Da-Ham-Sesu 2012}. Both examples include self-similar shrinkers (round circles for the former and constant curvature spheres for the latter) and a unique Type II ancient solution with curvature blowing up faster than $|t|$ as $t\to-\infty$ (the Angenent oval [paper clip] in the former case and the Rosenau in the latter). Our results here rule out the possibility of such Type II singularities and show that \emph{in any dimension}, the only weakly convex, ancient solutions are the ``self-similar" shrinkers: here, self-similar means conformally equivalent to the self-similar family of shrinking spheres in Euclidean space after realizing the sphere as (locally) conformally equivalent to Euclidean space.

In higher dimensions, classification results for ancient solutions in Euclidean space are less complete and a wider range of possibilities may occur. As mentioned above, the results of \cite{Hu-Sin 2014} require additional pinching assumptions to deduce that such ancient solutions are shrinking spheres. Wang \cite{Wang} studied translating ancient solutions and discovered the existence of non-rotationally symmetric translators (though they blow down to spheres or cylinders). Furthermore, there are solutions asymptotic to ``ovals" in that the center looks like a cylinder, but the ends look like ``bowls" \cite{Ang,Has-Her,Whi}. These solutions have similarities to the Angenent oval in the plane.

In the sphere, we may consider an ancient solution moving by a rotation to be a translating solution, but we quickly realize that no such convex solutions can exist because convexity and the containment principle force the flow to lie in a fixed hemisphere at all times. Our results also show that no non-self-similar ancient solutions exist. The phenomena of becoming more and more oval as $t\to -\infty$ for convex solutions appears to be ruled out by the compactness of the sphere; such a solution would have to touch an equator at the ``ends" and ``fatten" out in the middle, which forces it to touch the equator in the middle and thus everywhere (by convexity).

Lastly, the connection between Harnack inequalities, ancient solutions and solitons deserves mention. In Euclidean space, the classic work \cite{Hamilton 95} espouses the philosophy that Harnack inequality should be an equality on solitons. We have already observed that unlike Euclidean space, no translating convex solitons exist in the sphere. The equivalent of homothetic solitons may be taken to be those solutions of the mean curvature flow following the integral curves of the conformal "position" vector field $\sin(d)\nabla d$, where $d$ denotes the distance to a fixed point (the point of collapse). \textcolor[rgb]{1.00,0.00,0.00}{This vector field generates a one-parameter subgroup of conformal transformations, under which the mean curvature flow is not invariant, though the family of shrinking geodesic spheres is such a soliton and hence these conformal solitons are of great interest. The lack of conformal invariance suggests that conformal solitons do not satisfy equality in the Harnack inequality, and this is indeed the case for the shrinking spheres, so our results are perhaps not sharp. Interested readers should consult \cite{hun-nor 12,smo 97, smo 01} for more details on solitons interpreted as the flow along integral curves of a vector field, particularly \cite{arr-sun 13} for details on conformal solitons.}

\section*{Acknowledgement}
The first author would like to thank Bennett Chow for helpful discussions and encouragement. He would also like to thank the second author and the Institut f\"ur Diskrete Mathematik und Geometrie, TU Wien for hosting a very enjoyable and productive visit to Vienna. The coffee houses of Vienna most certainly experienced a downturn in business after his departure. The work of the second author was supported by Austrian Science Fund (FWF) Project M1716-N25.

\section{Preliminaries}
Let $g=\{g_{ij}\}$, $A=\{h_{ij}\},$ and $Rm_{ijkl}$ denote, in order, the induced metric, the second fundamental form, and the curvature tensor of $M^n$. The mean curvature of $M^n$ is the trace of the second fundamental form with respect to $g$, $H=g^{ij}h_{ij}.$ Let $\overline{g}$ and $\overline{Rm}_{\alpha\beta\gamma\theta}$ denote, respectively, the metric and the curvature tensor of $\mathbb{S}^{n+1}$. Greek indices run through $\{0,\cdots,n\}$ and Latin indices belong to the set $\{1,\cdots,n\}.$

Write $\nu$ for the outer unit normal to $M_t.$ For a fixed time, we choose a local orthonormal frame $\{\partial_0=\nu,\cdots,\partial_i=\frac{\partial F(\cdot,t)}{\partial x_i},\cdots,\partial_n\}$ in $\mathbb{S}^{n+1}.$ We use the following standard notation
\[h_i^j=g^{mj}h_{im}\]
\[(h^2)_i^j=g^{mj}g^{rs}h_{ir}h_{sm}\]
\[|A|^2=g^{ij}g^{kl}h_{ik}h_{lj}=h_{ij}h^{ij}\]
\[C=g^{ij}g^{kl}g^{mn}h_{ik}h_{lm}h_{nj}=h_i^kh_k^lh_l^i.\]
Here, $\{g^{ij}\}$ is the inverse matrix of $\{g_{ij}\}.$

The relations between $A$, $Rm$, and $\overline{Rm}$ are given by the Gau{\ss} and Codazzi equations:
\[Rm_{ijkl}=\overline{Rm}_{ijkl}+h_{ik}h_{jl}-h_{il}h_{jk}\]
\[\nabla_ih_{jk}=\nabla_{k}h_{ij}.\]
Moreover, $\nabla_i$ and $\Delta$ commute as follows
\[(\nabla_i\Delta-\Delta\nabla_i)f=-Rc_i^j\nabla_jf\]
for all smooth functions on $M^n.$
Therefore, in view of $\bar{R}_{\alpha\beta\gamma\theta}=\lambda (\overline{g}_{\alpha\gamma}\overline{g}_{\beta\theta}-\overline{g}_{\alpha\theta}\overline{g}_{\beta\gamma})$, $\lambda =1$, and the Gau{\ss} equation we have
\begin{equation}\label{eq: commute}
(\nabla_i\Delta-\Delta\nabla_i)H=((h^2)_i^m-Hh_i^m-(n-1)\delta_i^m\lambda )\nabla_mH.
\end{equation}
\section{Evolution equations}
In this section, we assume that $M_t$ is a strictly convex solution of the mean curvature flow.
\begin{lemma}\label{lem: lem3}
The following evolution equations hold under mean curvature flow:
\begin{enumerate}
  \item $\partial_tg_{ij}=-2Hh_{ij}$
  \item $\partial_tg^{ij}=2Hg^{im}g^{jn}h_{ij}=2Hh^{ij}$
  \item $\partial_t h_i^j=\nabla^j\nabla_iH+H(h^2)_i^j+\lambda H\delta_i^j$
  \item $\partial_t h_i^j=\Delta h_i^j+|A|^2h_i^j+\lambda \{2H\delta_i^j-nh_i^j\}$
  %\item $\partial_t h_{ij}=\nabla_j\nabla_iH-H(h^2)_{ij}+\lambda Hg_{ij}$
  \item $\partial_t h_{ij}=\Delta h_{ij}+|A|^2h_{ij}-2H(h^2)_{ij}+\lambda \{2Hg_{ij}-nh_{ij}\}$
  \item $\partial_t H=\Delta H+H|A|^2+n\lambda H$
  \item $(\partial_t\Delta-\Delta\partial_t)H=2Hh^{ij}\nabla_i\nabla_jH+2h^{ij}\nabla_iH\nabla_jH.$
\end{enumerate}
\end{lemma}
\begin{proof} The computations for $(1-6)$ are straightforward (see for example \cite{Huisken 87}). To obtain $(7)$, we note that
\begin{align*}
(\partial_t\Delta-\Delta\partial_t)H=\left(\partial_i\partial_jH-\Gamma_{ij}^k\partial_kH\right)\partial_tg^{ij}-g^{ij}\partial_kH\partial_t\Gamma_{ij}^k,
\end{align*}
where $\Gamma_{ij}^k=g^{kl}(\partial_ig_{jl}+\partial_jg_{il}-\partial_lg_{ij}).$ Since $\partial_t\Gamma_{ij}^k$ is tensorial, using $(1)$ and $ (2)$ we may carry out our calculations in a normal frame to prove the claim.
\end{proof}
\begin{lemma}\label{lem: lem1}
Under mean curvature flow we have
\begin{align*}
\partial_t(\partial_tH)=&
\Delta \partial_tH+4Hh^{ij}\nabla_i\nabla_jH+2h^{ij}\nabla_iH\nabla_jH\\
&+(|A|^2+n\lambda )(\partial_t H)+2H^2C+2\lambda H^3,
\end{align*}
and
\begin{align*}
\nabla_i\partial_tH
=\Delta\nabla_iH+\nabla_i(|A|^2H)+((h^2)_i^m-Hh_i^m)\nabla_mH+\lambda \nabla_iH.
\end{align*}
\end{lemma}
\begin{proof} Using $(3), (6)$ and $(7)$ in Lemma \ref{lem: lem1} we calculate
\begin{align*}
\partial_t(\partial_tH)=&\partial_t(\Delta H+H|A|^2+n\lambda H)\\
=&\Delta \partial_tH+2Hh^{ij}\nabla_i\nabla_jH+2h^{ij}\nabla_iH\nabla_jH\\
&+ (|A|^2+n\lambda )(\partial_t H)+H(\partial_t|A|^2)\\
=&\Delta \partial_tH+2Hh^{ij}\nabla_i\nabla_jH+2h^{ij}\nabla_iH\nabla_jH\\
&+ (|A|^2+n\lambda )(\partial_t H)+H(2h^{ij}\nabla_i\nabla_jH+2HC+2\lambda H^2)\\
=&\Delta \partial_tH+4Hh^{ij}\nabla_i\nabla_jH+2h^{ij}\nabla_iH\nabla_jH\\
&+(|A|^2+n\lambda )(\partial_t H)+2H^2C+2\lambda H^3.
\end{align*}
To obtain the second evolution equation, we use identity (\ref{eq: commute}) and part $(6)$ of Lemma \ref{lem: lem1}:
\begin{align*}
\nabla_i\partial_tH=&\nabla_i(\Delta H+|A|^2H+n\lambda H)\\
=&\Delta\nabla_iH+\nabla_i(|A|^2H)+n\lambda \nabla_iH\\
&+((h^2)_i^m-Hh_i^m-(n-1)\delta_i^m\lambda )\nabla_mH\\
=&\Delta\nabla_iH+\nabla_i(|A|^2H)+((h^2)_i^m-Hh_i^m)\nabla_mH+\lambda \nabla_iH.
\end{align*}
\end{proof}
In the sequel, $\{b^{ij}\}$ denotes the inverse of the second fundamental form.
\begin{lemma}\label{lem: lem 3}
\begin{align*}
\partial_t( b^{ij}\nabla_iH\nabla_jH)\leq&-b^{im}b^{jn}\Delta h_{mn}\nabla_iH\nabla_jH+2b^{ij}\nabla_jH\Delta\nabla_iH\\
&+2h^{ij}\nabla_iH\nabla_jH
+|A|^2b^{ij}\nabla_iH\nabla_jH+2Hb^{ij}\nabla_i|A|^2\nabla_jH\\
&+n\lambda b^{ij}\nabla_iH\nabla_jH.
\end{align*}
\end{lemma}
\begin{proof}
Observe
\begin{align*}
-\lambda b^{im}b^{jn}\{2Hg_{mn}-nh_{mn}\}\nabla_iH\nabla_jH+2\lambda b^{ij}\nabla_iH\nabla_jH\leq n\lambda b^{ij}\nabla_iH\nabla_jH
\end{align*}
and
\begin{align*}
\partial_t b^{ij}=&-b^{im}b^{jn}\partial_th_{mn}\\
=&-b^{im}b^{jn}(\Delta h_{mn}+|A|^2h_{mn}-2H(h^2)_{mn})\\
&-\lambda b^{im}b^{jn}\{2Hg_{mn}-nh_{mn}\}\\
=&-b^{im}b^{jn}\Delta h_{mn}-|A|^2b^{ij}+2Hg^{ij}\\
&-\lambda b^{im}b^{jn}\{2Hg_{mn}-nh_{mn}\}.
\end{align*}
Therefore, using the second part of Lemma \ref{lem: lem1} we obtain
\begin{align*}
\partial_t( b^{ij}\nabla_iH\nabla_jH)=&-b^{im}b^{jn}(\Delta h_{mn}+|A|^2h_{mn}-2H(h^2)_{mn})\nabla_iH\nabla_jH\\
&-\lambda b^{im}b^{jn}\{2Hg_{mn}-nh_{mn}\}\nabla_iH\nabla_jH\\
&+2b^{ij}(\Delta\nabla_iH+\nabla_i(|A|^2H)+((h^2)_i^m-Hh_i^m)\nabla_mH+\lambda \nabla_iH)\nabla_jH\\
\leq& -b^{im}b^{jn}\Delta h_{mn}\nabla_iH\nabla_jH+2b^{ij}\nabla_jH\Delta\nabla_iH\\
&+2b^{im}b^{jn}H(h^2)_{mn}\nabla_iH\nabla_jH\\
&+|A|^2b^{ij}\nabla_iH\nabla_jH+2Hb^{ij}\nabla_i|A|^2\nabla_jH\\
&+2b^{ij}((h^2)_i^m-Hh_i^m)\nabla_mH\nabla_jH\\
&+n\lambda b^{ij}\nabla_iH\nabla_jH\\
=&-b^{im}b^{jn}\Delta h_{mn}\nabla_iH\nabla_jH+2b^{ij}\nabla_jH\Delta\nabla_iH\\
&+2h^{ij}\nabla_iH\nabla_jH
+|A|^2b^{ij}\nabla_iH\nabla_jH+2Hb^{ij}\nabla_i|A|^2\nabla_jH\\
&+n\lambda b^{ij}\nabla_iH\nabla_jH.
\end{align*}
\end{proof}
Using the identities
\[\nabla_mb^{ij}=-b^{ik}b^{jl}\nabla_mh_{kl}\]
and
\begin{align*}
\Delta b^{ij}=&-b^{im}b^{jn}\Delta h_{mn}\\
&+\{b^{ir}b^{ks}b^{jl}+b^{ik}b^{jr}b^{ls}\}g^{pq}\nabla_ph_{rs}\nabla_qh_{kl},
\end{align*}
we compute
\begin{align*}
\Delta( b^{ij}\nabla_iH\nabla_jH)=&-b^{im}b^{jn}\nabla_iH\nabla_jH\Delta h_{mn}+2b^{ij}\nabla_jH\Delta\nabla_iH\\\
&+2\{b^{ir}b^{ks}b^{jl}\}\nabla_iH\nabla_jHg^{pq}\nabla_ph_{rs}\nabla_qh_{kl}\\
&-4g^{mn}b^{ik}b^{jl}\nabla_mh_{kl}\nabla_n\nabla_iH\nabla_jH+2b^{ij}g^{mn}(\nabla_m\nabla_iH\nabla_n\nabla_jH).
\end{align*}
Thus we have proved:
\begin{lemma}\label{lem: lem2}
\begin{align*}
\partial_t(b^{ij}\nabla_iH\nabla_jH)\leq &\Delta (b^{ij}\nabla_iH\nabla_jH)\\
&-2\{b^{ir}b^{ks}b^{jl}\}\nabla_iH\nabla_jHg^{pq}\nabla_ph_{rs}\nabla_qh_{kl}\\
&+4g^{mn}b^{ik}b^{jl}\nabla_mh_{kl}\nabla_n\nabla_iH\nabla_jH-2b^{ij}g^{mn}(\nabla_m\nabla_iH\nabla_n\nabla_jH)\\
&+2h^{ij}\nabla_iH\nabla_jH+|A|^2b^{ij}\nabla_iH\nabla_jH+2Hb^{ij}\nabla_i|A|^2\nabla_jH\\
&+n\lambda b^{ij}\nabla_iH\nabla_jH.
\end{align*}
\end{lemma}
\begin{lemma}\label{lem: lema6}
Define $\Theta:=\partial_tH-b^{ij}\nabla_iH\nabla_jH$. Then
\begin{align*}
\partial_t\Theta&\geq\Delta \Theta+ \frac{2(\Theta-n\lambda H)^2}{H}+(|A|^2+n\lambda )\Theta\\
&+2\{g^{mq}b^{np}-\frac{g^{mn}g^{pq}}{H}\}\eta_{mn}\eta_{pq}+2\lambda H^3,
\end{align*}
where
\[\eta_{mn}=\nabla_m\nabla_nH+H(h^2)_{mn}-b^{rs}\nabla_rH\nabla_sh_{mn}.\]
\end{lemma}
\begin{proof}
Note that
\begin{align*}
2\frac{(\Theta-n\lambda H)^2}{H}=&2\frac{(\Delta H)^2}{H}+4|A|^2\Delta H-4\frac{\Delta H}{H}b^{ij}\nabla_iH\nabla_jH\\
&+2|A|^4H-4|A|^2b^{ij}\nabla_iH\nabla_jH\\
&+\frac{2}{H}\left(b^{ij}\nabla_iH\nabla_jH\right)^2
\end{align*}
and
\begin{align*}
2\{g^{mq}b^{np}&-\frac{g^{mn}g^{pq}}{H}\}\eta_{mn}\eta_{pq}\\
=&2g^{mq}b^{np}\nabla_m\nabla_nH\nabla_p\nabla_qH-2\frac{(\Delta H)^2}{H}\\
&+4Hh^{ij}\nabla_i\nabla_jH-4|A|^2\Delta H\\
&-4g^{mq}b^{np}b^{rs}\nabla_m\nabla_nH\nabla_rH\nabla_sh_{pq}+4\frac{\Delta H}{H}b^{ij}\nabla_iH\nabla_jH\\
&+2H^2C-2|A|^4H\\
&-2Hb^{ij}\nabla_iH\nabla_j|A|^2+4|A|^2b^{ij}\nabla_iH\nabla_jH\\
&+2\{b^{ir}b^{ks}b^{jl}\}\nabla_iH\nabla_jHg^{pq}\nabla_ph_{rs}\nabla_qh_{kl}-\frac{2}{H}\left(b^{ij}\nabla_iH\nabla_jH\right)^2.
\end{align*}
Now the claim follows from adding up these last two identities and considering Lemmas \ref{lem: lem1} and \ref{lem: lem2}.
\end{proof}
\section{Harnack estimate and Backwards Convergence}
\subsection{Harnack estimate}
If $M_0$ is not an equator, the strong parabolic maximum principle and the evolution equation of $h_i^j$ in Lemma \ref{lem: lem1} implies that for any $t>0$, $M_t$ is strictly convex. For strictly convex hypersurfaces, $$A(V,V)+2D_VH$$ is minimized by $V=(V^1,\cdots,V^i=-b^{ij}\nabla_iH,\cdots,V^n)$ and so to prove the main theorem, it suffices to verify that for all $t>0$
\[\partial_tH-b^{ij}\nabla_iH\nabla_jH-n\lambda H +\frac{H}{2t}\geq 0.\]
Fix $0<\varepsilon<T.$ We will apply the maximum principle to
$$Q:=\frac{\Theta-n\lambda H}{H}=\frac{\Delta H+|A|^2H-b^{ij}\nabla_iH\nabla_jH}{H}$$
on the time interval $[\varepsilon, T).$ Using Lemmas \ref{lem: lem3}, \ref{lem: lema6}, and inverse-concavity of the mean curvature we calculate
\begin{align*}
\partial_tQ=\partial_t \frac{\Theta}{H}\geq&\Delta Q+2\langle \nabla Q,\frac{\nabla H}{H}\rangle+ 2Q^2+2\lambda H^2\\
&+2\frac{\{g^{mq}b^{np}-\frac{g^{mn}g^{pq}}{H}\}\eta_{mn}\eta_{pq}}{H}\\
\geq& \Delta Q+2\langle \nabla Q,\frac{\nabla H}{H}\rangle+ 2Q^2.
\end{align*}
An ODE comparison with $q(t)=-\frac{1}{2(t-\varepsilon)}$ which satisfies $\frac{d}{dt}q(t)=2q^2(t)$ and $\lim\limits_{t\to \varepsilon^+}q(t)=-\infty$ shows that $Q(\cdot,t)\geq q(t)$ for any $t>\varepsilon.$ Letting $\varepsilon\to 0$ completes the proof of the Harnack estimate.
%\begin{theorem}[Harnack inequality] Suppose $M_t$ is a strictly convex solution of the mean curvature flow. Suppose $t_2>t_1>0,$ then
%\[\frac{H(x_2,t_2)}{H(x_1,t_1)}\geq \left(\frac{t_1}{t_2}\right)^{\frac{1}{2}}\exp{\left(-\frac{d^2(x_1,x_2,g(t_1))}{4(t_2-t_1)}\right)}.\]
%\end{theorem}
%\begin{proof}
%We can integrate over paths in space-time to get this inequality, see proof of Theorem 5.17 of "Harnack inequalities for evolving hypersurfaces" by Andrews.
%\end{proof}
\subsection{Backwards convergence}
\begin{lemma}\label{cor: backward limit of principal curvatures}
Any weakly convex ancient solution of the mean curvature flow satisfies
\[|A|\leq c_0\exp(nt)\]
for $t\le 0$
for some $c_0$ depending only on $M_0.$
\end{lemma}
\begin{proof}
By the strong  parabolic maximum principle, any weakly convex ancient solution of the mean curvature flow must become strictly convex, unless it is a non-moving equator. By Theorem 11 for strictly convex, ancient solutions we have
\[\frac{\partial_tH-b^{ij}\nabla_iH\nabla_jH-n\lambda H}{H}\geq 0.\]
Since $\lambda =1$, we get
\[\partial_t\log H\geq n.\]
Integrating both sides of this inequality against $dt$ on the time interval $[t,0]$ gives
\[H(\cdot,t)\leq H(\cdot,0)\exp(nt).\]
\end{proof}
\begin{lemma}
Any weakly convex ancient solution of the mean curvature flow satisfies
\[|\nabla^mA|^2\leq c_m\exp(2nt)\]
for $t\le 0,$ where $c_m$ depends only on $M_0$ and $m.$
\end{lemma}
\begin{proof}
The proof follows by induction on $m.$
Using the fourth evolution equation in Lemma \ref{lem: lem3}, $\partial_t\Gamma_{ij}^k=A\ast\nabla A$ and that the commutator $[\nabla^k,\Delta]T $ is given by
\[[\nabla^k,\Delta]T =\sum\limits_{j=0}^k\nabla^jRm\ast\nabla^{k-j}T\] for any tensor $T$, we can compute the following evolution equations:
\begin{enumerate}
  \item \[\partial_t |A|^2=\Delta |A|^2-2|\nabla A|^2+2|A|^4+2\lambda(2H^2-n|A|^2)\]
  \item \begin{align*}
  \partial_t |\nabla^mA|^2=&\Delta |\nabla^mA|^2-2 |\nabla^{m+1}A|^2\\
  &+\sum\limits_{i+j+k=m}\nabla^iA\ast\nabla^jA\ast\nabla^kA\ast\nabla^mA\\
  &+\lambda\nabla^mA\ast\nabla^mA.
         \end{align*}
\end{enumerate}
On the other hand, by Lemma \ref{cor: backward limit of principal curvatures} we get
\[\partial_t |A|^2\leq\Delta |A|^2-2|\nabla A|^2+c_1\exp(2nt),\]
and
\[\partial_t |\nabla A|^2\leq\Delta |\nabla A|^2+c_2|\nabla A|^2+c_3\exp(2nt).\]
Here, $c_1,c_2,c_3\geq0$ are independent of $t.$ Therefore,
\begin{align*}
\partial_t ((t-s)|\nabla A|^2+b|A|^2)\leq&\Delta((t-s)|\nabla A|^2+b|A|^2)\\
&+(1-2b+(t-s)c_2)|\nabla A|^2+(c_1b+c_3)\exp(2nt).
\end{align*}
We want to apply the maximum principle on the time interval $[s,s+1].$
We choose $b$ large enough, independent of $t$, so that the coefficient of $|\nabla A|^2$ becomes negative. Thus the maximum principle implies that
\[(t-s)|\nabla A|^2\leq c_4\exp(2n t)(1-\exp(2n(s-t) ))+bc_0^2 \exp(2ns).\]
In particular for $t=s+1$ we have
\[|\nabla A|^2(\cdot, t)\leq c_5\exp(2n t).\]
This verifies the bound on $|\nabla A|.$ Higher derivative estimates follow by induction and using the test function
$(t-s)|\nabla^mA|^2+b_{m-1}|\nabla^{m-1}A|^2:$
\[\partial_t |\nabla^{m-1} A|^2\leq\Delta |\nabla^{m-1} A|^2-2 |\nabla^{m}A|^2+c_1\exp(2nt).\]
\[\partial_t |\nabla^m A|^2\leq\Delta |\nabla^m A|^2+c_2|\nabla^m A|^2+c_3\exp(2nt).\]
\end{proof}
Having established the higher derivative curvature bounds, convexity of $M_t$ implies that the backwards limit of $M_t$ is independent of subsequences. Therefore we have proved:
\begin{theorem}\label{thm8}
Let $M_t$ be an ancient, embedded, weakly convex solution of mean curvature flow. Then $M_t$ converges exponentially fast in $C^{\infty}$ to a unique equator $M_{-\infty}$ as $t\to-\infty.$
\end{theorem}


\section{Classification of Ancient Solutions}

In this section, we use Theorem \ref{thm8} to classify convex, embedded ancient solution of the mean curvature flow on \(\sphere^{n+1}\). The proof uses the Aleksandrov reflection as in \cite{Br-Lou}.

We begin with some preliminaries of the Aleksandrov reflection on \(\sphere^{n+1}\). First, we will work relative to the limiting equator obtained in Theorem \ref{thm8}. Let \(E\) be an equator that bounds the \emph{open} hemispheres \(H^{\pm}\) with centers \(\pm \basepoint\), and let \(\vertvec = \overrightarrow{\origin\basepoint}\) be the unit vector in \(\RR^{n+2}\) that points from the origin \(\origin\) to \(\basepoint\). Let \(\radialdistance(x) = d_{\sphere^{n+1}} (\basepoint, x)\) denote the spherical distance from \(\basepoint\) to \(x \in \sphere^{n+1}\). The radial projection onto \(E\) is the map \(x \in \sphere^{n+1} \mapsto \radialprojection(x) \in E\), where \(\radialprojection\) is the nearest point on \(E\) to \(x\). If \(x \ne \pm \basepoint\), then \(\radialprojection(x)\) is a single point. If \(x = \pm \basepoint\), then \(\radialprojection(x) = E\). In any event, given \(y \in \radialprojection(x)\), there is a unique length minimizing geodesic joining \(x\) to \(y\) and this geodesic must pass through \(\pm \basepoint\).

It is convenient to make use of the ambient \(\RR^{n+2}\) and define the height function \(\height(x) = \ip{x}{\vertvec}\). The radial distance is related to the height function via
\[
\height(x) = \cos(\radialdistance(x))
\]
which is monotonically decreasing in \(\radialdistance\).

Now for the Aleksandrov reflection, let \(\reflectionvector \in \RR^{n+2}\) be any unit vector that \(\ip{\reflectionvector}{\vertvec} < 0\). Let \(\reflectionplane = \reflectionvector^{\perp}\) be the hyperplane through the origin with the normal vector \(\reflectionvector\). Let \(\reflectionhalfspace^{\pm} = \{\pm \ip{x}{\reflectionvector} > 0\}\) denote the halfspaces with the boundary \(\reflectionplane\). For any subset \(S \subset \sphere^{n+1}\), write \(\reflectionset{S}^{\pm} = S \intersect \reflectionhalfspace^{\pm}\). Lastly, let \(\delta > 0\) denote the angle \(\reflectionvector\) makes with \(E\); therefore, \(\sin \delta = \ip{\reflectionvector}{-\vertvec}\).

\begin{definition}
The Aleksandrov reflection across \(\reflectionplane\) is the map defined by
\[
\reflectionmap: x \in \RR^{n+1} \mapsto x - 2\ip{x}{\reflectionvector} \reflectionvector.
\]
\end{definition}

This map is an (orientation reversing) isometry of \(\RR^{n+2}\) fixing \(\reflectionplane\) and in particular fixing the origin. Therefore, it induces an isometry of \(\sphere^{n+1}\). For \(x \in E\), we have \(\ip{x}{\vertvec} = 0\) and
\[
\height(\reflectionmap(x)) = \ip{\vertvec}{x - 2 \ip{x}{\reflectionvector} \reflectionvector} = 2 \sin\delta \ip{x}{\reflectionvector}.
\]
In the case \(x \in E^+\), \(\height(\reflectionmap(x)) > 0\), and in the case \(x \in E \intersect \reflectionplane\), \(\height(\reflectionmap(x)) = 0\).

In geodesic polar coordinates, \((\radialdistance, \sigma) \in (0, \pi) \times E \simeq \sphere^{n+1}\backslash \{\pm \basepoint\}\), a smooth, closed hypersurface \(M \subset \reflectionhalfspace^+\) that bounds a region in \(\reflectionhalfspace\) is a smooth graph \((f(\sigma), \sigma)\) over \(E\) if and only if its (outer) normal \(\nor\) satisfies \(\ip{\nor}{\vertvec} < 0\) (e.q. \(M\) has no vertical tangents). In particular, for all \(\epsilon>0\) there is a \(\xi>0\) such that if \(M\) is a graph with \(\ip{\nor}{\vertvec} < -\xi\) and \(N\) is \(\epsilon\)-close to \(M\) in \(C^{\infty}\), then \(N\) is a graph over \(E\).

\begin{theorem}
Let \(M_t\) be a convex, embedded ancient solution of the mean curvature flow on \(\sphere^{n+1}\). Then \(M_t\) is a family of shrinking geodesic spheres emanating from an equator at \(t=-\infty\).
\end{theorem}

\begin{proof}
Let \(E = M_{-\infty} \simeq \sphere^n\) be the limiting equator \(\lim\limits_{t\to-\infty} M_t\).

Since \(M_t\) smoothly converges to \(E\) as \(t\to-\infty\), we may write \(M_t\) as the graph of a smooth positive function over \(E\) in the geodesic polar coordinates, \(M_t = \{(f_t(\sigma), \sigma) \in (0,\pi) \times E\)\}. We have \(f_t \to \pi/2\) smoothly and uniformly in \(C^{\infty}\) as \(t \to -\infty\). Moreover, for \(\delta \in (0,\pi/4)\), \(\reflectionmap(E) = \{(g_{-\infty}(\sigma), \sigma)\)\} is a graph over \(E\).

We want to send \(\delta \to 0\) eventually. Let us fix a \(\delta_0 \in (0,\pi/4)\) to give us a little room because \(\reflectionmap(E)\) is not a graph when \(\delta = \pi/4\) (it is an equator perpendicular to \(E\)). Explicitly, given any \(\delta \in (0,\delta_0)\), there exists a \(C_{\delta} > 0\), such that if \(\|f_t - \pi/2\|_{C^{\infty}} < C_{\delta}\), then \(\reflectionmap(M_t)\) is a graph over \(E\) with \(C^{\infty}\)-norm bounded by \(C_{\delta}\). Consequently, since \(f_t \to \pi/2\) smoothly, there is a \(T_{\delta} < 0\) such that for every \(t \leq T_{\delta}\), \(\reflectionmap(M_t) =\{ (g_t(\sigma), \sigma)\)\} is a graph.

As noted above, \(\height(\reflectionmap(x)) > 0\) on \(E^+\) and \(\height(x) = \cos(\radialdistance(x))\). Thus, continuity implies that for \(\epsilon > 0\) there exits an \(\eta>0\) such that \(\radialdistance(\reflectionmap(x)) < \pi/2 - \epsilon\) provided \(x \in E_{\eta} = \{x \in E: d(x, E \intersect \reflectionplane) > \eta\}\). By making \(T_{\delta}\) smaller, since \(M_t \to_{C^{\infty}} E\), we can arrange that \(d(M_t, E) < \epsilon/2\) for all \(t < T_{\delta}\); that is, \(\radialdistance (x) > \pi/2 - \epsilon/2\) for all \(x \in M_t\). Now for \(x \in M_t^+ \intersect \radialprojection^{-1} E_{\eta}\), since \(\reflectionmap\) is an isometry, we have \(d(\reflectionmap(x), \reflectionmap(\radialprojection(x))) < \epsilon/2\); therefore, \(\radialdistance(\reflectionmap(x)) < \pi/2 - \epsilon/2\). Consequently, away from the strip \(\{x \in E: d(x, E \intersect \reflectionplane) \leq \eta\}\), we have \(\radialdistance(\reflectionmap(\reflectionset{(M_t)}^+)) < \pi/2 - \epsilon/2\) and \(\radialdistance(M_t^-) > \pi/2 - \epsilon/2\). That is, away from the strip, we have \(\reflectionmap(\reflectionset{(M_t)}^+) > \reflectionset{(M_t)}^-\).

Now let \(C\subset E\) be any great circle and consider \(f_t|_C\). Using the Backwards Approximate Symmetry Lemma \cite[Lemma 5.1]{Br-Lou} (which applies whenever \(f_t\) converges smoothly to \(C\)), we find that (possibly by decreasing \(T_{\delta}\)) over \(C\)
\[
\reflectionmap(\reflectionset{(M_t)}^+) \geq \reflectionset{(M_t)}^-
\]
on the strip \(\{x \in E: d(x, E \intersect \reflectionplane) \leq \eta\}\). Here, \(T_{\delta}\) depends only on \(\|f_t - \pi/2\|_{C^{\infty}}\) and \(\delta\); therefore, \(T_{\delta} > - \infty\) is independent of \(C\). Thus we find that
\begin{equation}
\label{eq:backwards_approximate_symmetry}
\reflectionmap(\reflectionset{(M_t)}^+) \geq \reflectionset{(M_t)}^-
\end{equation}
everywhere for all \(t \in (-\infty, T_{\delta})\). 

\textcolor[rgb]{1.00,0.00,0.00}{Let us now define \(T_{\delta}\) so that \((-\infty, T_{\delta})\) is the largest interval on which relation \eqref{eq:backwards_approximate_symmetry} holds. Also define \(T = \inf\limits_{\delta \in (0,\delta_0)} T_{\delta}\). We want to show that \(T > -\infty\), and hence that the relation \eqref{eq:backwards_approximate_symmetry} holds on the non-empty, open interval \((-\infty, T)\). To show $T>-\infty$, we apply the maximum principle as in \cite[Lemma 5.2]{Br-Lou}: We know that \(\reflectionset{(M_t)}^-\) and \(\reflectionmap(\reflectionset{(M_t)}^+)\) lie in the interior of \(\reflectionhalfspace^-\) with the common boundary lying in \(\reflectionplane \intersect \sphere^{n+1}\). We also know that in a neighborhood of \(\reflectionplane \intersect \sphere^{n+1}\), the hypersurfaces \(\reflectionset{M_t}^-\) and \(\reflectionmap(\reflectionset{(M_t)}^+)\) are disjoint for \(t\) sufficiently negative (depending on \(\delta\)) because \(\reflectionset{M_t}^-\) is \(C^1\)-close to the equator, while \(\reflectionmap(\reflectionset{(M_t)}^+)\) is \(C^1\)-close to the reflected equator. The strong maximum principle, using a parabolic version of the Hopf boundary point lemma (\cite{Chow 97}), ensures that the relation \(\reflectionmap(\reflectionset{(M_t)}^+) \geq \reflectionset{(M_t)}^-\) is preserved along the flow as long as both \(\reflectionset{M_t}^-\) and \(\reflectionmap(\reflectionset{(M_t)}^+)\) are non-empty and intersect \(\reflectionplane\) transversely. These latter conditions are true on an open interval \((-\infty, T)\) with \(T>-\infty\) independent of \(\delta\) as in \cite[Lemma 5.2]{Br-Lou}. Therefore, since \eqref{eq:backwards_approximate_symmetry} implies this relation is true on \((-\infty, T_{\delta})\), the maximum principle then ensures that
\begin{equation*}
\label{eq:longtime_approximate_symmetry}
\reflectionmap(\reflectionset{(M_t)}^+) \geq \reflectionset{(M_t)}^-
\end{equation*}
for all \(t \in (-\infty, T)\) and any \(\delta \in (0,\delta_0)\).}

To complete the proof, we use \cite[Proposition 5.3]{Br-Lou} (which applies in any dimension) to conclude that \(M_t\) is a geodesic sphere for all \(t \in (-\infty, T)\) and thus for all negative times by the uniqueness of solutions.
\end{proof}


\bibliographystyle{amsplain}
\begin{thebibliography}{10}
\bibitem{Ang} S. Angenent, ``Formal asymptotic expansions for symmetric ancient ovals in mean curvature flow." Networks and Heterogeneous Media \textbf{8}(2013) 1--8.
\bibitem{Andrews 94} B Andrews, ``Harnack inequalities for evolving hypersurfaces." Mathematische Zeitschrift \textbf{217}(1994): 179--197.
\bibitem{arr-sun 13} C. Arezzo and J. Sun, ``Conformal solitons to the mean curvature flow and minimal submanifolds." Mathematische Nachrichten \textbf{286}(2013): 772-790.
\bibitem{Br-Lou} P. Bryan and L. Janelle, ``Classification of convex ancient solutions to curve shortening flow on the sphere." Journal of Geometric Analysis, (2015) DOI: 10.1007/s12220-015-9574-x.
\bibitem{Hamilton 95} R.S. Hamilton, ``Harnack estimate for the mean curvature flow." Journal of Differential Geometry \textbf{41}(1995): 215--226.
\bibitem{Chow 97} B. Chow, ``Geometric aspects of Aleksandrov reflection and gradient estimates for parabolic equations." Communications in Analysis and Geometry \textbf{5}(1997): 389--409.
\bibitem{Chow-Gul 01} B. Chow and R. Gulliver, ``Aleksandrov reflection and geometric evolution of hypersurfaces." Communications in Analysis and Geometry \textbf{9}(2001): 261--280.
\bibitem{Chow-Gul 96} B. Chow and R. Gulliver, ``Aleksandrov reflection and nonlinear evolution equations. I. The $n$-sphere and $n$-ball." Calculus of Variations and Partial Differential Equations \textbf{4}(1996): 249--264.
\bibitem{Da-Ham-Sesu 2010} P. Daskalopoulos, R. Hamilton and N. Sesum, ``Classification of compact ancient solutions to the curve shortening flow." Journal of Differential Geometry \textbf{84}(2010): 455--464.
\bibitem{Da-Ham-Sesu 2012} P. Daskalopoulos, R. Hamilton and N. Sesum, ``Classification of ancient compact solutions to the {R}icci flow on surfaces." Journal of Differential Geometry \textbf{91}(2012): 171--214.
\bibitem{Huisken 87} G. Huisken, ``Deforming hypersurfaces of the sphere by their mean curvature." Mathematische Zeitschrift \textbf{195}(1987): 205--219.
\bibitem{Hu-Sin 1999} G. Huisken and C. Sinestrari, ``Mean curvature flow singularities for mean convex surfaces." Calculus of Variations and Partial Differential Equations \textbf{8}(1999): 1-14.
\bibitem{Hu-Sin 2014} G. Huisken and C. Sinestrari, ``Convex ancient solutions of the mean curvature flow." arXiv preprint arXiv:1405.7509 (2014).
\bibitem{Has-Her} R. Haslhofer and O. Hershkovits, ``Ancient solutions of the mean curvature flow." arXiv preprint arXiv:1308.4095 (2013).
\bibitem{hun-nor 12} N. Hungerb{\"u}hler and T. Mettler, ``Soliton solutions of the mean curvature flow and minimal hypersurfaces." Proceedings of the American Mathematical Society \textbf{140}(2012): 2117--2126.
\bibitem{smo 97} K. Smoczyk, ``Harnack inequalities for curvature flows depending on mean curvature." New York Journal of Mathematics \textbf{3}(1997): 103--118.
\bibitem{smo 01} K. Smoczyk, ``A relation between mean curvature flow solitons and minimal submanifolds." Mathematische Nachrichten \textbf{229}(2001): 175--186.
\bibitem{Wang} X-J.Wang, ``Convex solutions to the mean curvature flow." Annals of Mathematics. Second Series \textbf{173}(2011): 1185--1239.
\bibitem{Whi} B. White, ``The nature of singularities in mean curvature flow of mean convex sets." Journal of the American Mathematical Society \textbf{16}(2003): 123--138.
\bibitem{Urbas 91} J.I.E. Urbas, ``An expansion of convex hypersurfaces." Journal of Differential Geometry \textbf{33}(1991): 91--125.
\end{thebibliography}
\end{document}
